\documentclass[a4paper]{article}

\usepackage[english]{babel}
\usepackage[utf8]{inputenc}
\usepackage{amsmath}
\usepackage{graphicx}
\usepackage[colorinlistoftodos]{todonotes}
\usepackage{fancyhdr}

\fancypagestyle{plain}{%
\fancyhf{}
\fancyhead[LO,RE]{\sffamily\bfseries\large}
\fancyhead[RO,LE]{\sffamily\bfseries\large }
\fancyfoot[LO,RE]{\sffamily\bfseries\large }
\fancyfoot[RO,LE]{\sffamily\bfseries\thepage}
\renewcommand{\headrulewidth}{0pt}
\renewcommand{\footrulewidth}{0pt}
}

\pagestyle{fancy}
\fancyhf{}
\fancyhead[RO,LE]{\sffamily\bfseries\large Grid and Cloud Computing}
\fancyhead[LO,RE]{\sffamily\bfseries\large Paper review 1}
\fancyfoot[LO,RE]{\sffamily\bfseries\large }
\fancyfoot[RO,LE]{\sffamily\bfseries\thepage}
\renewcommand{\headrulewidth}{1pt}
\renewcommand{\footrulewidth}{0pt}

\newcommand{\paper}{Finding a needle in Haystack: Facebook's photo storage}

\title{Grid and Cloud Computing \\ \paper}

\author{Nicky Advokaat, Tim van Dalen}

\date{\today}

\begin{document}

\maketitle

\begin{abstract}
In this paper we will review the paper \paper \cite{thepaper}.  
\end{abstract}

\section{Abstract of the paper}
	The paper describes Haystack, a new architecture/infrastructure setup that Facebook is using to store their users' photos.
	Haystack improves on the old system by reducing the number of I/O lookups needed to process a single photo request.
	The authors explain why content delivery networks are not very useful in solving this problem.
	Haystack consists of three major components: the directory, the cache and the store.

\section{Main strenghts}
  \begin{itemize}
	  \item With Facebook's growth, such a system is definitely needed. Since the paper was published, the estimated number of photos uploaded to Facebook has gone from 1 billion to 2.5 billion per week.
	  \item The paper shows solid experiments.
    \item Since photos are very important to Facebook's business, we know the authors were really driven to design a good system.
	  \item Haystack is in use today. The theoretical aspects in the paper are put to the test every day.
  \end{itemize}

\section{Main weaknesses}
	\begin{itemize}
	  \item While the system is explained well, most details are proprietary, so it's hard to reproduce the paper.
	  \item The same goes for the comparison with the old system.
	\end{itemize}

\begin{thebibliography}{9}

\bibitem{thepaper}
  Doug Beaver, Sanjeev Kumar, Harry C. Li, Jason Sobel, Peter Vajgel,
  \emph{\paper},
  Facebook Inc, 2010.

 
\end{thebibliography}

\end{document}
